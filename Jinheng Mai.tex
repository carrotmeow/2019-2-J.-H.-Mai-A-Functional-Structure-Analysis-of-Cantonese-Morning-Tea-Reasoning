%!TEX encoding=UTF-8 Unicode
%%%%%%%%%%%%%%%%%%%%%%%%%%%%%%%%%%%%%%%%%%%%%%%%%%%%%%%%%%%%%%%%%%%%%%%%%%%
%%%%%%%%%%%%%%%%%%%%%%%%%%%%%%%%%%%%%%%%%%%%%%%%%%%%%%%%%%%%%%%%%%%%%%%%%%%
%%                                                                       %%
%%                    《逻辑学研究》中文论文模板                         %%
%%                                                                       %%
%%              中山大学逻辑与认知研究所逻辑学研究编辑部                 %%
%%                                                                       %%
%%                             Ver 1.31                                  %%
%%                                                                       %%
%%        You can modify it and distribute it freely    2014.06.04       %%
%%                                                                       %%
%%%%%%%%%%%%%%%%%%%%%%%%%%%%%%%%%%%%%%%%%%%%%%%%%%%%%%%%%%%%%%%%%%%%%%%%%%%

%-------------------------------------------------------------------------%
%
%  请以第一作者全拼为文件名另存此文档(后缀名仍为.tex),与 SLCN.sty 保存
%
%  在同一个文件夹中。你可能需要取消某些行首的注释符以添加需要的内容。
%  使用XeLaTex编译
%  参考文献的排版,请作者创建 .bib 文件, 并使用 BibTex 或者 Biber(有中文参考文献时) 进行排版。
%-------------------------------------------------------------------------%

%=========================================================================%
%                        一、编辑处理部分
%
%                  *** 作者请直接跳至第二部分 ***
%=========================================================================%

%-------------------------------------------------------------------------%
%    1.1 设定纸张大小、正文字体大小
%-------------------------------------------------------------------------%
\documentclass[b5paper,10.5pt,onecolumn,twoside,leqno,UTF8]{article}
\usepackage{SLCN}                                                         % 加载版面格式
\usepackage{bussproofs}
\setmainfont{Times New Roman}

%-------------------------------------------------------------------------%
%    1.2 填入卷期号、出版年月
%-------------------------------------------------------------------------%
\newcommand{\myvolnumber}{12}                                              % 输入卷号
\newcommand{\myissnumber}{2}                                              % 输入当年期号
\newcommand{\mypubyear}{2019}                                             % 输入出版年份

%-------------------------------------------------------------------------%
%    1.3 填入起止页码、页数
%-------------------------------------------------------------------------%
\newcommand{\myfirstpage}{1}                                              % 输入起始页码
\newcommand{\mylastpage}{10}                                              % 输入终止页码
\newcommand{\mypages}{10}                                                 % 输入页数

%-------------------------------------------------------------------------%
%    1.4 填入收稿日期、修改稿日期
%-------------------------------------------------------------------------%
\newcommand{\receiveddate}{2018-10-31}                                    % 输入本文收稿日期
\newcommand{\revisiondate}{null}                                          % 预置修订日期为空,勿改此行
%\renewcommand{\revisiondate}{xxxx-xx-xx}                                 % 输入修订日期(若有)并取消该行注释

%-------------------------------------------------------------------------%
%    1.5 填入作者、单位英译名
%-------------------------------------------------------------------------%
\newcommand{\mysecondauthorEN}{null}                                      % 预置第二作者为空,请不要修改此行
\newcommand{\mythirdauthorEN}{null}                                       % 预置第三作者为空,请不要修改此行
\newcommand{\myfourthauthorEN}{null}
\newcommand{\myfifthauthorEN}{null}

\newcommand{\myfirstauthorEN}                                             % 请输入第一作者姓名
{Jinheng Mai}

\newcommand{\myfirstaffiliationEN}{%% 请输入第一作者单位
{School of Sociology and Anthropology, Sun Yat-sen University}
}

%\renewcommand{\mysecondauthorEN}{jiangSecond Author}                               % 若需要,请输入第二作者姓名,并取消该行注释
\newcommand{\mysecondaffiliationEN}{                                      % 若需要,请输入第二作者单位
{Department of Philosophy, Sun Yat-sen University}
}%School of Philosophy and Social Development, Shandong University

%\renewcommand{\mythirdauthorEN}{Third Author}                                % 若需要,请输入第三作者姓名,并取消该行注释
\newcommand{\mythirdaffiliationEN}{                                       % 若需要,请输入第三作者单位
{Department of Philosophy, Sun Yat-sen University} 
}

%\renewcommand{\myfourthauthorEN}{Fourth Author}
\newcommand{\myfourthaffiliationEN}{
{Department of Philosophy, Sun Yat-sen University}
}


%\renewcommand{\myfifthauthorEN}{Fifth Author}
\newcommand{\myfifthaffiliationEN}{
{Department of Philosophy, Sun Yat-sen University}
}

%-------------------------------------------------------------------------%
%    1.6 填入文章类型
%(original, bookreview, conferencereport)默认为original
%-------------------------------------------------------------------------%
\newcommand{\myarticletype}
{original}

\newcommand{\reviewbooktitle}                                             % 若文章为书评,请输入所评书的出版信息
{The information of the book reviewed by the author}

\newcommand{\reviewbooktitleEN}{null}                                     % 预置所评书的中译版为空,请勿修改此行
%\renewcommand{\reviewbooktitleEN}{中译版信息}                            % 若书有中译版,请输入中译版信息并取消该行注释

%-------------------------------------------------------------------------%
%    1.7 填入责任编辑
%-------------------------------------------------------------------------%
\newcommand{\myeditor}{%
{\bf (责任编辑:罗心澄)}
}

%-------------------------------------------------------------------------%
%    1.8 缺省设置
%-------------------------------------------------------------------------%
\newcommand{\mysecondauthor}{null}                                        % 预置第二作者为空,请不要修改此行
\newcommand{\mythirdauthor}{null}                                         % 预置第三作者为空,请不要修改此行
\newcommand{\mygrants}{null}                                              % 预置“项目资助”为空,请不要修改此行
\newcommand{\mythanks}{null}                                              % 预置“致谢”为空,请不要修改此行

\usepackage{graphicx}
%=========================================================================%
%
%                        二、作者写作部分
%
%=========================================================================%

%-------------------------------------------------------------------------%
%    2.1 请填入论文题目、作者姓名、单位、电子邮箱
%-------------------------------------------------------------------------%
\newcommand{\mytitle}                                                     % 请输入论文题目
{广式早茶说理的功能结构分析\\
\Large——以“詏”为例
}

\newcommand{\myrunningtitle}                                              % 请输入用于页眉的标题(可能需要缩短原来的标题)
{广式早茶说理的功能结构分析}

\newcommand{\myfirstauthor}                                               % 请输入第一作者姓名
{\kaishu 麦劲恒}

\newcommand{\myfirstaffiliation}                                          % 请输入第一作者单位,多个单位用 \\\small 分隔
{中山大学社会学与人类学学院}%\\
%\small 阳明大学~心智哲学研究所}

\newcommand{\myfirstemail}%% 请输入第一作者电子邮箱
{dante\_mai@163.com}

%\renewcommand{\mysecondauthor}{\kaishu 第三}                                 % 若需要,请输入第二作者姓名,并取消该行注释

\newcommand{\mysecondaffiliation}                                         % 若需要,请输入第二作者单位,多个单位用\\\small分隔
{ 单位2,单位 2-2}

\newcommand{\mysecondemail}                                               % 若需要,请输入第二作者邮箱
{xxxx@xxxx.xxx2}

%\renewcommand{\mythirdauthor}{\kaishu 第三作者}                                  % 若需要,请输入第三作者姓名,并取消该行注释

\newcommand{\mythirdaffiliation}                                          % 若需要,请输入第三作者单位,多个单位用\\small分隔
{单位3(到系所)}

\newcommand{\mythirdemail}                                                % 若需要,请输入第三作者邮箱
{xxxx@xxxx.xxx3}

\newcommand{\myfourthauthor}{null}

%\renewcommand{\myfourthauthor}{\kaishu 第四作者}
\newcommand{\myfourthaffiliation}{ 单位四}

\newcommand{\myfourthemail}
{xxxx@xxxx.xxx4}

\newcommand{\myfifthauthor}{null}

%\renewcommand{\myfifthauthor}{\kaishu 江璐}
\newcommand{\myfifthaffiliation}{单位五}

\newcommand{\myfifthemail}
{xxxx@xxxx.xxx5}
%-------------------------------------------------------------------------%
%    2.2 请填入项目资助、致谢(选填)
%-------------------------------------------------------------------------%
\renewcommand{\mygrants}{本文受教育部人文社会科学重点研究基地重大项目“广义论证理论的研究”资助(16JJD720017)。}                  % 若需要,请输入项目名称及批号,并取消该行注释,多个项目以中文逗号分隔
\renewcommand{\mythanks}{感谢鞠实儿老师的悉心指导。}                          % 若需要,请输入致谢内容,并取消该行注释

%-------------------------------------------------------------------------%
%    2.3 请填入中文摘要、关键词
%-------------------------------------------------------------------------%
\newcommand{\myabstract}                                                  % 请在下面输入中文摘要
{地方性说理作为一种常见的日常交际,其规则的研究一直是各个学科的研究焦点之一,但是长久以来未有一套很好的研究方法。常人方法学、会话分析和互动社会语言学虽然为地方性说理研究提供了崭新的常人方法视角,但是他们的方法未能很好地提取地方性说理规则。本文试图运用广义论证与功能分析法对广式早茶中的说理案例进行分析。结果表明,该方法有着较高适用性与可操作性,能够系统地提取地方性说理规则。}

\newcommand{\mykeywords}                                                  % 请在下面输入中文关键词,以中文分号分隔
{说理; 广义论证; 功能分析; 广式早茶;“詏”}%逻辑史;辩证学;《名理探》;科因布拉注疏系列}说理  广义论证  功能分析  广式早茶  “詏”

%-------------------------------------------------------------------------%
%    2.4 请填入英文标题、摘要(默认为null)
%-------------------------------------------------------------------------%
\newcommand{\mytitleEN}                                                   % 请输入英文标题
{A Functional Structure Analysis of Cantonese \\ Morning Tea Reasoning
\\ \large --- Taking ``\emph{Au}'' for Example}

\newcommand{\myabstractEN}                                                % 请输入英文摘要
{As a common daily communication,
 the study of local reasoning rules has always been one of the research focuses of various disciplines,
but for a long time there has not been a good set of research methods. 
Although ethnomethodology, CA and interactional sociolinguistics provided a new ethno-method perspective for the study of local reasoning, 
their method have not been able to extract local reasoning rules very well. 
This paper attempts to analyze the case of `\emph{au}' argumentation in Cantonese morning tea by means of general argumentation and functional analysis. 
The result show that the method has high applicability and maneuverability, and can extract local reasoning rules systematically.
}

%-------------------------------------------------------------------------%
%    2.5 预置宏包和自定义命令(你可以补充需要的宏包和自定义命令)
%-------------------------------------------------------------------------%
\usepackage[pdfborder=0,CJKbookmarks=true]{hyperref}  % 使用内部超链接,其中第二个选项用于支持中文书签

\addbibresource{Jinheng Mai.bib}				%填入参考文献库文件 .bib

\newcommand{\pcr}[1]{\raise0.1em\hbox{\parencite{#1}}}
\newcommand{\prc}{\pcr}%[1]{\raise0.1em\hbox{\parencite{#1}}}
\newcommand{\cpcr}[1]{(\pcr{#1})}
 
\newcommand{\cndash}{\raise.1em\hbox{--}}
\newcommand{\pc}[2]{(\pcr{#1},第 #2 页)}

\newcommand{\hto}{\mathrm{H}_2\mathrm{O}}

\usepackage{latexsym}


\usepackage{indentfirst}
\usepackage[all]{xy}

\makeatletter
\renewenvironment{quotation}
               {\renewcommand{\baselinestretch}{1.2} \it
               \list{}{\listparindent 2em%
               \rightmargin=0em \leftmargin=2em
                        \itemindent    \listparindent
                        \rightmargin   \leftmargin
                        \parsep        \z@ \@plus\p@}%
                \item\relax}
               {\endlist}
               
\renewenvironment{quote}
               {\renewcommand{\baselinestretch}{1.2} \it
               \list{}{\listparindent 0em%
               \rightmargin=0em \leftmargin=2em
                        \itemindent    \listparindent
                        \rightmargin   \leftmargin
                        \parsep        \z@ \@plus\p@}%
                \item\relax}
               {\endlist}
               
\makeatother   


%-------------------------------------------------------------------------%
%    2.6 打印标题页信息(作者请忽略此部分)
%-------------------------------------------------------------------------%
\begin{document}
\begingroup                                                               % 以下使得收稿日期以不加标记的脚注出现
\makeatletter
\let\@makefnmark\relax\footnotetext{%
  \ifthenelse{\equal{\revisiondate}{null}}{{\bf 收稿日期:}\receiveddate}{
    {\bf 收稿日期:}\receiveddate;{\bf 修订日期:}\revisiondate
  }
}
\makeatother
\endgroup
\printtitlepage                                                           % 打印标题、作者、摘要等信息


  

%-------------------------------------------------------------------------%
%    2.7 正文内容从这里开始
%-------------------------------------------------------------------------%


\section{导言}

交际互动的规则一直是社会学、逻辑学、人类学等学科研究的焦点之一。以韦伯(M.~Weber)和帕森斯(T.~Parsons)为代表的经典社会学尝试以抽象的概念和模型来把握社会现实中稳定的本质。上世纪五六十年代兴起的常人方法学和会话分析认为这与事实不符\pc{gar}{6、23} ,他们认为人们是有自己的方法——亦即常人方法——来实现交际的有序性和融洽性的,而不是死板地遵循抽象的、外在的规则来行事的。互动社会语言学是基于会话分析、交际民族志等学科发展而来的\pc{gum}{161} ,这个学科同样反对传统语言学和社会学对抽象概念和结构的研究,主张研究日常的交际互动,关注特定的文化情境中人们相互理解的机制。上述研究方法的兴起使得基于常人方法视角的经验性研究成为可能。近年来,广义论证的提出\cpcr{ju} ,使得人们开始将注意力转移至特定的文化框架下的说理研究中,关注日常的说理活动。从这个角度上看,广义论证也是基于常人方法视角的,但是广义论证有着一套更系统的研究程序和功能分析法,从而能够更有效地发现系统的本土说理规则。


广式早茶作为广州人重要的休闲与交际空间,有着其独特文化内涵和交际功能。街坊、亲朋甚至互不相识的人都可以在早茶情境中交换信息、交流情感、发表观点,这体现着广州人休闲的生活方式和开放包容的文化特征。广式早茶中的说理是基于上述文化氛围产生的一种典型的地方说理交际,“詏”是其中一种以粤语语篇展开的论证形式,其功能便是说理。从粤语的角度解释,“詏”有着争论的意思\pc{mai}{207} ,随着茶客们“谈天说地”,基于不同观点的“詏”时有发生。但是大家面红耳赤地“詏”完之后,明天总能开开心心地在茶桌上相聚,这正因为广式早茶的情境赋予“詏”特殊的意义和功能。本文在对比上述研究方法的基础上,尝试以广义论证的功能分析法来分析广式早茶中的“詏”,从而验证其适用性和可行性。




\section{问题与方法}

在地方性说理研究领域,研究者首要面对的问题就是人们在说理的过程中遵循着什么样的规则?以什么方法来研究这些规则?本研究也不例外。互动社会语言学与常人方法学、会话分析有着密切的关系,他们为解答我们上述所关注的问题作出了一定的贡献。但这些方法能否直接套用至地方性说理研究中?这是需要探讨的。

\subsection{常人方法学与会话分析}

加芬克尔(H.~Garfinkel)开创的常人方法学反对传统社会学以抽象的概念和结构来解释行为的做法,他们主张回归到实践现象的本身\pc{gar}{1--2} 。他们认为人们在互动的过程中,规则是通过行为的不断的生产与再生产建构出来的,并不是外在于行为的抽象物,规则只有在行为本身中寻找。基于上述思想,基于常人方法的经验性研究是可能的而且是很必要的,这能够使研究者更加贴近实践现象的本身,发现更真实的规则。萨克斯(H.~Sacks)的会话分析正是基于该思想发展而来的对“实际发生事件的细节”观察的社会学\pc{sac}{26} 。会话分析立足于日常会话,关注会话中的社会秩序以及人们达成这些秩序的方法\pc{lyn}{24--25} 。其一般的研究方法是通过录音录像等手段对日常会话进行收集,然后对这些录音录像进行精细转写,进而提取规则\cpcr{yu}。

常人方法学与会话分析的兴起使得基于常人方法的经验性研究成为可能,但常人方法学与会话分析拒斥抽象概念,因此其关于地方性的规则的描述往往过于琐碎而缺乏概括性。对此,互动社会语言学有进一步的发展。

\subsection{互动社会语言学}

针对会话分析的不足,互动社会语言学致力于对交际事件中人们相互理解的线索\cpcr{gum4} 和“交际框架”(frame of interaction)的把握\pc{gum}{167} 。

互动社会语言学认为,交际的核心是意图\pc{zheng}{48--49} ,为了刻画交际意图,互动社会语言学者着重描写会话推理和策略\pc{gum3}{215} 。在交际的过程中,参与者必须依靠各种语言、副语言和非语言信息来揣摩对方的意图,继而选择交际策略,这些表面信息被称为语境化线索(contextualization cues)。语境化线索的使用与理解遵循特定的地方性规约,遵守相同语境规约的人群能够在交际中对正常的节奏、音高、语调、语体等因素有一定的预设和期待,同时也能够识别这些线索。

互动社会语言学的另一个关注点是“交际框架”。在一次成功的交谈中,“交际框架”实质上是与人们所使用的相同的说话模式、节奏和相近的语境化线索等因素相关的。人们通过语境化线索相互交流和理解的过程中,他们会渐渐达成关于在该次交际中该说什么、如何说等问题的共识,从而实现交际的“同步性”,使会话变得流畅\pc{gum}{167} 。达成上述“同步性”的过程实质上就是“交际框架”商议的过程\pc{gum}{167} 。我们可以认为,“交际框架”是研究者们大概地确定事件类型、分析事件中的主要交际情境和交际线索的重要依据。目前,互动社会语言学的方法已经被广泛运用到单语、双语甚至多语互动的研究中\cpcr{gum2} 。

互动社会语言学在确定特定类型的交际事件方面有了长足的进步,但是他们依然无法帮助研究者发现系统性的规则。到底以什么因素为切入点来描述一个“交际框架”?人们在某种地方性说理事件中到底遵循着一种怎么样的行为模式?这些问题尚待解决。

\subsection{广义论证与功能分析法}

广义论证是隶属于一个或多个社会文化群体的成员(即论证参与者),在相应的社会文化背景下、依据所属社会文化群体的规范下生成的一个基于语篇的社会互动序列。其目的是:劝使论证参与者对有争议的观点或论点采取某种态度,消除分歧从而达成一致意见\cpcr{ju2} 。广义论证认为,论证的过程是在特定文化规范下实现意图的过程,该过程包括以下几个步骤:(1) 由目标意图来决定语效意图;(2) 由语效意图来决定语旨行为类型;(3)由语旨行为类型来确定具体话语\cpcr{ju1} 。

为了还原上述意图实现的过程并且发现系统的说理规则,广义论证的办法是在本土文化的语境下对说理交际进行系统的语篇和功能分析\cpcr{ju2} 。鞠实儿认为,人们在特定文化情境的说理过程中,为了实现目的,会以语篇的形式展开论证。每个子语篇均能实现相应的行为功能,众多子语篇组成的总语篇为说理者的终极目标服务。在这个过程中,说理者会产生一个行为功能结构与对应的语篇结构。在实现特定的目标时,功能结构是相对稳定的,而语篇结构是可变的,与说理者的行为偏好、语境、文化等因素有关。

为了发现上述的功能结构与语篇结构,鞠实儿提出了一套系统的研究程序,他称之为“五步法”\cpcr{ju2} 。第一步,研究者需进入田野整理本土文化中关于论证的相关背景知识,如语言、信仰、价值、宗教信念、社会制度、文化习俗等。第二步,研究者需进入田野广泛收集论证的经验数据。第三步,用会话分析、语用学、功能分析等方法分析数据,在此基础上归纳出候选的逻辑规则,亦即上述的功能结构和语篇结构。第四步,运用第一步所得的文化背景知识对候选规则进行合理性说明。第五步,重回田野,运用第四步所得的规则进行论证实践,对它们进行归纳检验,通过检验者为该文化中控制论证进程的逻辑规则。

该研究程序的优点是:基于常人方法的视角,从实际的说理交际中发现系统的地方性规则,确保科学归纳和验证的严谨性的同时,不否认规则的灵活性和内生性。本研究试图采用上述的研究程序的前四步来发现广式早茶中具有说理功能的“詏”式论证的功能结构,从而验证该研究程序的适用性和可操作性。

\section{广式早茶“詏”式论证功能结构分析}

基于广义论证的研究思路,我们可以在广式早茶的文化语境下对其中的“詏”进行较深入的研究。在研究的第一步,我们有必要对广式早茶中的“詏”文化背景进行了解,这包括早茶交际情境和话题、人际关系、“詏”的发生条件和基本特征等。

\subsection{广式早茶“詏”的背景知识}

到广式茶楼饮早茶是广州人悠久的文化习俗之一,人们的饮茶场所从清末作为苦力、商贩歇脚之地的“二厘馆”形态发展到今天的各式茶楼酒楼,历久不衰\pc{mo}{1--14} 。虽然大家交流的信息有所改变,但其仍然是广州市民亲朋聚会、交流感情、传播信息的重要地方。在茶楼情境下,人们无所不谈,上至国家大事、经济民生,下至家头细务、妯娌琐事、街坊传闻等均可以作为讨论话题,总之广州人喜闻乐见的闲聊都可以在早茶中找到。

广式早茶交际情境的核心特征之一是“搭台”,或称“孖台”。茶楼很多时用的是大茶桌,人们允许其他人一起来搭台饮茶,即使不认识的人可以坐一桌饮茶甚至聊天,一则新闻或者一味点心都能让大家聊起来。随着近年来茶楼的更多元化发展,茶楼店面越来越大,某些茶楼既可以提供小桌让亲戚朋友们“约茶”小聚,同时也提供大圆桌让各式各样的茶客搭台,很多“老广”每天最期盼的活动便是与老街坊老茶友们搭台聊聊家常。虽然广式茶楼中也不乏亲戚、朋友之间的“约茶”,但广式早茶的交际有别于其他餐饮场所的私密化交际,“搭台”这一习俗催生了广州特有的开放交际空间和交际情境。在这种情境下,具有地方特色的说理便随之产生。

在广式早茶的情境下,人们基于不同的人际关系交流信息,那么在此基础上,人们会出于各种目的运用各种策略使得对方接受自己的观点,说理便随之而来,“詏”便是其中一种能够实现说理功能的典型论证形式。如前文所述,粤语中的“詏”是争论的意思,“詏”有时会伴有一些情感的宣泄,乃至“发抆憎”\footnote{因懊恼而恶言相向或黑脸相向。\pc{zhang}{93}}。人们“詏”的时候可能会用一些广式粗口、情绪化策略,因此“詏”往往带有些许广式市井味道。在搭台现象还算盛行的当代广式茶楼中,茶客们的背景各异,大家交际起来难免简单粗暴。这种搭台关系所产生的情景往往能够促使这种论证的产生,当大家持有不同观点的时候,大家难免会“詏”。

需要指出的是,虽然在其他情境中人们也会“詏”,但滋生在早茶情境下的“詏”则有着独特的交际功能。广式早茶楼情景是相对轻松、开放的,人们“詏”的时候一般不抱有敌意,也不必分出胜负。这样的说理既能满足茶客们抒发观点、宣泄感情、建立威信等欲望,也能给予大家足够的面子。因此,茶客们的关系可以得到维持乃至加深,他们“詏”过以后也会继续搭台。总的来说,广式早茶中的“詏”多数是为了沟通感情而发生的。

\subsection{“詏”案例分析与功能结构提取}
在充分了解文化背景知识的前提下,我们开始执行研究的第二和第三步,那就是收集案例,然后对语篇策略进行功能分析,进而提取功能结构。以下是一个“詏”案例分析示例。

生活在五羊邨的两位老人家CE(87岁,女)和XL(91岁,男)是多年的老街坊,而且通过长期的搭台成了好茶友。XL是码头工人出身,年幼时读过几年“卜卜斋(私塾)”,在他们那个年代也算是文化人。而CE是“疍家人”,从小没有机会读书,因此目不识丁。自恃着肚子里有点墨水的XL每次跟CE聊天的时候总想着彰显自己的文化,而CE却恰好不吃这一套,结果他们经常都会“詏”。

某日早晨,XL“呻”(吐槽)了让他很郁闷的一件家庭琐事,他每天早上都有用耳机听粤曲的习惯,结果那天却被妻子说XL听粤曲吵着她了,XL觉得耳机造成的噪音实在有限,因此抱怨妻子无理取闹。CE开始持反对意见,后来被XL的观点说服。

\begin{table}[h!]
\caption{案例:“家吵屋闭”}

\end{table}

一开始CE运用斩钉截铁的语调提出广州人众所周知的道理试图让XL息怒,这激发起广州人关于 “家和万事兴,家衰口不停”的一般共识。这些语境化线索结合广州人共享的认知使得CE的发言具有相当的力度。然而CE建立的权威并没有如愿地让XL息怒,因为XL本想通过“呻”来宣泄郁闷,CE的话语令XL很扫兴,于是XL想跟CE“詏”一番。他在2--4行明确表明了反对态度,对此,CE采取解释观点的策略来为自己辩护,结果引来了XL的取笑(11--12行、14--16行)。

对此,CE依然想尽力避免冲突,她在第13行“即係我噉同你讲”这句话在粤语说理中非常常见,能够表达出自己本无敌意,只是表达观点罢了。可见这句话本应能够息事宁人,以缓和对方的敌意从而维护自己的权威和面子。但XL的回应使该功能落空了,XL在18行进一步实施了取笑策略。极端重读与拉长配合降调的“啊”是重要的线索,在粤语语境下该线索有揭短的意味和功能,在这里就引出了对CE“懂得说别人却不懂得说自己”的讽刺。

XL的策略成功激怒了CE,她针对XL观点的适用性进行还击,她认为两夫妻之间的争吵会导致“家衰”。但CE的这个观点被XL所利用,XL将这个观点往CE所期待的相反方向解释,他认为迁就会使一方被欺负这也会导致“家衰”。这个过程中,双方均想实现驳倒对方与辩护自己观点两大行为功能。

XL的策略非常凑效,他对上述共同认识的另类诠释引起了CE的认同。CE的态度开始变得没这么强硬,她在27行的通俗化表达为XL提供了一个权宜之策。其意思是自己顾自己,不乱生事端,这是本次会话进入结论阶段的重要转折点。随后,CE配合XL完成总结性的论证,在有保留地认同XL的基础上(37行带有语调波动的“係”),提出了相对折中的结论(39--40行)。CE这两个策略能够表达自己认同XL的胜利地位,同时为自己争取面子,实现双方的关系平衡,最终实现维护茶客关系这一功能。这时胜负已经很明显了,如果XL还继续针对CE的观点的话,那就实在太不识趣了,XL也深知这一点。XL和CE是茶楼里的长者,他们需顾及对方的面子,将胜负的事实心照不宣,折中的观点或者转换话题便是最好的解决办法,他们一般会乐意接受。上述CE和XL的行为功能与语篇策略如下图所示:

\begin{figure}[h!]

\caption{CE的行为功能与语篇策略示意图}
\end{figure}

\begin{figure}[h!]


\caption{XL的行为功能与语篇策略示意图}
\end{figure}
从上述可见,“詏”的一级行为功能为“有效地提出观点”、“反对与反驳对方”、“支撑与辩护自己的观点”和“维护茶客间的关系”,这些功能下面会有若干二级功能,这些功能有助于实现上述的一级功能。我们可以通过更多的案例分析来修正和补充上述功能结构,从而完成第三步研究程序。


\subsection{功能结构的合理性说明}

在第一步研究程序的背景知识的基础上,我们可以进行第四步研究程序——对功能结构进行合理性说明。“詏”是茶客们维系和增进感情的重要手段,我们虽然在上述功能结构中发现一般论辩的“反对与反驳对方”和“支撑与辩护自己的观点”等行为功能,但是分胜负往往并不是他们的首要目的。因此,广式早茶中的“詏”往往是以搁置争议为终点,当实现了自己诸如抒发观点、调侃他人、建立威信等个人欲望之后,他们也会适可而止。需指出的是,广式早茶中的“詏”最重要的行为功能是“维护茶客间的关系”,正因为大家都有意识地实现这一功能,人们才会日复一日地在茶桌上相聚,在面红耳赤的争论中不断增进感情。因此,我们可以说上述的功能结构在早茶情境中是合理的。

\section{总结与展望}
本研究是广义论证功能分析法的一次重要实践,基于功能分析的思想和“五步法”研究程序,我们发现了广式早茶说理中的一些规则。在这个过程中,我们验证了功能分析法在地方性说理研究中的适用性与可操作性。相对于常人方法学、会话分析和互动社会语言学,广义论证倾向于以功能结构的形式描述地方性说理的规则,这种做法有如下优势:(1) 规则的呈现更直观,能够为外地人融入该文化情境提供有力的参考;(2) 以行为功能为切入点对复杂多变的言语交流进行分析,这使得分析的数据更具有可概括性,从而使成果更具系统性;(3) 兼顾相对稳定的功能结构与相对灵活的语篇策略,使所发掘的规则更贴近实践;(4) 有较强的可验证性和可修改性,研究者可以根据直观的行为功能和语篇策略,不断地通过新的案例分析来验证和修正自己的研究成果。同时,这也体现了规则的灵活性和内生性。

“五步法”研究程序为实现功能结构分析提供了有力的支撑,这种“从田野中来,到田野中去”的做法能够确保所得规则的地方性,摆脱抽象概念与固化的研究框架对事实的扭曲同时,保持科学归纳的严谨性。但是,本研究未能呈现“五步法”的第五步。第五步的实施方式可以很多种,如问卷法、参与观察法、交际实验法等等,这些方法各有不同的优缺点。今后的研究中,我们将继续探讨这些方法对于成果验证的适用性,进一步完善广义论证“五步法”研究程序。





%-------------------------------------------------------------------------%
%    2.8 参考文献
%-------------------------------------------------------------------------%
\vspace{2ex}
\printbibliography
%为了使参考文献按作者姓氏的拼音排列,使用biblatex里面的caspervector样式,需要使用biber编译,
%编译顺序是XeLatex -> biber -> XeLatex -> XeLatex。

%-------------------------------------------------------------------------%
%    2.9 打印作者信息
%-------------------------------------------------------------------------%
%\vspace*{4ex}\noindent\printauthors
%\ifthenelse{\equal{\myarticletype}{original}}{}{%
%\vspace*{4ex}
%\noindent{\kaishu \myfirstauthor}                                       % 第一作者
%{\myfirstaffiliation  }                                                   % 第一作者单位
%{\myfirstemail   }                                                          % 第一作者email
% \vspace*{1ex}                                                                %如果需要请取消注释
% {\kaishu \mysecondauthor}                                            % 第二作者
% {\mysecondaffiliation }                                                 % 第二作者单位
%{ \mysecondemail }                                                     % 第二作者email
% \vspace*{1ex}                                                           %如果需要请取消注释
% {\kaishu \mythirdauthor}                                              % 第三作者
% \mythirdaffiliation                                                   % 第三作者单位
% \mythirdemail                                                           % 第三作者email
%}

%-------------------------------------------------------------------------%
%    2.10 打印责任编辑(作者请忽略此部分)
%-------------------------------------------------------------------------%
\vspace{2ex}
\begin{flushright}
\myeditor
\end{flushright}

%-------------------------------------------------------------------------%
%    2.11 根据需要打印英文摘要(作者请忽略此部分)
%-------------------------------------------------------------------------%


\ifthenelse{\equal{\mytitleEN}{null}}{}{%
    \newpage
    \printtitlepageEN
}

\end{document}
